% !TEX root = Projektdokumentation.tex

% Es werden nur die Abkürzungen aufgelistet, die mit \ac definiert und auch benutzt wurden. 
%
% \acro{VERSIS}{Versicherungsinformationssystem\acroextra{ (Bestandsführungssystem)}}
% Ergibt in der Liste: VERSIS Versicherungsinformationssystem (Bestandsführungssystem)
% Im Text aber: \ac{VERSIS} -> Versicherungsinformationssystem (VERSIS)

% Hinweis: allgemein bekannte Abkürzungen wie z.B. bzw. u.a. müssen nicht ins Abkürzungsverzeichnis aufgenommen werden
% Hinweis: allgemein bekannte IT-Begriffe wie Datenbank oder Programmiersprache müssen nicht erläutert werden,
%          aber ggfs. Fachbegriffe aus der Domäne des Prüflings (z.B. Versicherung)

% Die Option (in den eckigen Klammern) enthält das längste Label oder
% einen Platzhalter der die Breite der linken Spalte bestimmt.
\begin{acronym}[WWWWWWWW]
	\acro{AGB}{Allgemeine Geschäftsbedingungen}
	\acro{AJAX}{Asynchronous JavaScript and XML}
	\acro{API}{Application Programming Interface}
	\acro{App}{Applikation}
	\acro{CSS}{Cascading Style Sheets}
	\acro{DBMS}{Datenbank-Management-System}
	\acro{DOM}{Document Object Model}
	\acro{EPK}{Ereignisgesteuerte Prozesskette}
	\acro{ERM}{En\-ti\-ty-Re\-la\-tion\-ship-Mo\-dell}
	\acro{FTP}{File Transfer Protocol}
	\acro{Git}{Versionskontrollsystem}
	\acro{GitHub}{Online-Dienst zur Verwaltung quelloffener Software}
	\acro{HR}{Human Resources/Personalabteilung}
	\acro{HTML}{Hypertext Markup Language}\acused{HTML}
	\acro{KPI}{Key Performance Indicator}
	\acro{LAN}{Local Area Network}
	\acro{MBO}{Merchant Backoffice}
	\acro{Mock}{Platzhalter für echte Daten}
	\acro{MongoDB}{Dokumentenorientierte NoSQL-Datenbank}
	\acro{MVC}{Model View Controller}
	\acro{MySQL}{Open-Source-Relationales-Datenbank-Verwaltungssystem}
	\acro{NodeJS}{serverseitige Plattform zur Softwareentwicklung mit Javascript}
	\acro{PC}{Personal Computer}
	\acro{PHP}{Hypertext Preprocessor}
	\acro{QA}{Quality Assurance}
	\acro{REST}{Representational state transfer}
	\acro{REST-API}{REST-Schnittstelle}
	\acro{SCP}{Secure Copy}
	\acro{Scrum}{Vorgehensmodell zur agilen Softwareentwicklung}
	\acro{SFTP}{SSH File Transfer Protocol}
	\acro{SQL}{Structured Query Language}
	\acro{SSH}{Secure Shell}
	\acro{UML}{Unified Modeling Language}
	\acro{URL}{Uniform Resource Locator (Webadresse)}
	\acro{WLAN}{Wireless Local Area Network}
\end{acronym}
