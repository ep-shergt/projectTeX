% !TEX root = ../Projektdokumentation.tex
\section{Entwurfsphase} 
\label{sec:Entwurfsphase}

\subsection{Zielplattform}
\label{sec:Zielplattform}

Die Statistik-App wird als Web-Applikation entwickelt und sollte damit lauffähig in allen gängigen Internetbrowsern eines jeden Heim-PCs oder Laptops sein, mindestens jedoch im Internet-Explorer, Firefox und Google-Chrome. Die App ist über eine feste URL auf dem \gqq{uberspace.de}-Server erreichbar. Die Wahl auf \gqq{uberspace} als Host fiel aufgrund der geringen monatlichen Mietkosten für das Hosting und die Servernutzung und der Unterstützung von MySQL. Die verwendeten Programmiersprachen umfassen Javascript und PHP. Dies ist der Tatsache geschuldet, dass viele meiner Kollegen und mein Ausbilder ein großes Know-How in diesen Sprachen haben und mich so effektiv unterstützen können. Des Weiteren ist für mich als Auszubildender in der Frontendentwicklung Javascript die Programmiersprache, in der ich mich am besten auskenne. Die Wahl auf PHP als serverseitige Sprache fiel aufgrund der großen Community im Vergleich zu NodeJS und der einfachen Datenbankanbindung an eine MySQL-Datenbank. NodeJS wird zwar für Single-Page-Anwendungen empfohlen, jedoch sollte man für rechenintensive Anwendungen, wie die Statistik-App eine ist, dann doch eher auf PHP zurückgreifen. Zudem gibt es von ePages auf der App-Entwicklungs-Webseite Tutorials zur App-Entwicklung in PHP und auch bereits eine eigene PHP-Klasse zur Anforderung von REST-Daten. Prinzipiell hätte man für diese Anwendung auch MongoDB als Datenbanksystem nehmen können. Das Vorhaben wurde aber mangels Erfahrung im Umgang damit verworfen.

\begin{itemize}
	\item Beschreibung der Kriterien zur Auswahl der Zielplattform (\ua Programmiersprache, Datenbank, Client/Server, Hardware).
\end{itemize}


\subsection{Architekturdesign}
\label{sec:Architekturdesign}
\begin{itemize}
	\item Beschreibung und Begründung der gewählten Anwendungsarchitektur (\zB \acs{MVC}).
	\item \Ggfs Bewertung und Auswahl von verwendeten Frameworks sowie \ggfs eine kurze Einführung in die Funktionsweise des verwendeten Frameworks.
\end{itemize}

\paragraph{Beispiel}
Anhand der Entscheidungsmatrix in Tabelle~\ref{tab:Entscheidungsmatrix} wurde für die Implementierung der Anwendung das \acs{PHP}-Framework Symfony\footnote{\Vgl \citet{Symfony}.} ausgewählt. 

\tabelle{Entscheidungsmatrix}{tab:Entscheidungsmatrix}{Nutzwert.tex}


\subsection{Entwurf der Benutzeroberfläche}
\label{sec:Benutzeroberflaeche} 
\begin{itemize}
	\item Entscheidung für die gewählte Benutzeroberfläche (\zB GUI, Webinterface).
	\item Beschreibung des visuellen Entwurfs der konkreten Oberfläche (\zB Mockups, Menüführung).
	\item \Ggfs Erläuterung von angewendeten Richtlinien zur Usability und Verweis auf Corporate Design.
\end{itemize}

\paragraph{Beispiel}
Beispielentwürfe finden sich im \Anhang{app:Entwuerfe}.


\subsection{Datenmodell}
\label{sec:Datenmodell}

\begin{itemize}
	\item Entwurf/Beschreibung der Datenstrukturen (\zB \acs{ERM} und/oder Tabellenmodell, \acs{XML}-Schemas) mit kurzer Beschreibung der wichtigsten (!) verwendeten Entitäten.
\end{itemize}

\paragraph{Beispiel}
In \Abbildung{ER} wird ein \ac{ERM} dargestellt, welches lediglich Entitäten, Relationen und die dazugehörigen Kardinalitäten enthält. 

\begin{figure}[htb]
\centering
\includegraphicsKeepAspectRatio{ERDiagramm.pdf}{0.6}
\caption{Vereinfachtes ER-Modell}
\label{fig:ER}
\end{figure} 


\subsection{Geschäftslogik}
\label{sec:Geschaeftslogik}

\begin{itemize}
	\item Modellierung und Beschreibung der wichtigsten (!) Bereiche der Geschäftslogik (\zB mit Kom\-po\-nen\-ten-, Klassen-, Sequenz-, Datenflussdiagramm, Programmablaufplan, Struktogramm, \ac{EPK}).
	\item Wie wird die erstellte Anwendung in den Arbeitsfluss des Unternehmens integriert?
\end{itemize}

\paragraph{Beispiel}
Ein Klassendiagramm, welches die Klassen der Anwendung und deren Beziehungen untereinander darstellt kann im \Anhang{app:Klassendiagramm} eingesehen werden.

\Abbildung{Modulimport} zeigt den grundsätzlichen Programmablauf beim Einlesen eines Moduls als \ac{EPK}.
\begin{figure}[htb]
\centering
\includegraphicsKeepAspectRatio{modulimport.pdf}{0.9}
\caption{Prozess des Einlesens eines Moduls}
\label{fig:Modulimport}
\end{figure}


\subsection{Maßnahmen zur Qualitätssicherung}
\label{sec:Qualitaetssicherung}
\begin{itemize}
	\item Welche Maßnahmen werden ergriffen, um die Qualität des Projektergebnisses (siehe Kapitel~\ref{sec:Qualitaetsanforderungen}: \nameref{sec:Qualitaetsanforderungen}) zu sichern (\zB automatische Tests, Anwendertests)?
	\item \Ggfs Definition von Testfällen und deren Durchführung (durch Programme/Benutzer).
\end{itemize}


\subsection{Pflichtenheft/Datenverarbeitungskonzept}
\label{sec:Pflichtenheft}
\begin{itemize}
	\item Auszüge aus dem Pflichtenheft/Datenverarbeitungskonzept, wenn es im Rahmen des Projekts erstellt wurde.
\end{itemize}

\paragraph{Beispiel}
Ein Beispiel für das auf dem Lastenheft (siehe Kapitel~\ref{sec:Lastenheft}: \nameref{sec:Lastenheft}) aufbauende Pflichtenheft ist im \Anhang{app:Pflichtenheft} zu finden.


\Zwischenstand{Entwurfsphase}{Entwurf}
