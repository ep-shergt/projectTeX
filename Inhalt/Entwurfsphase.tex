% !TEX root = ../Projektdokumentation.tex
\section{Entwurfsphase} 
\label{sec:Entwurfsphase}

\subsection{Zielplattform}
\label{sec:Zielplattform}

Die Statistik-App wird als Web-Applikation entwickelt und sollte damit lauffähig in allen gängigen Internetbrowsern eines jeden Heim-PCs oder Laptops sein, mindestens jedoch im Internet-Explorer, Firefox und Google-Chrome. Die App ist über eine feste URL auf dem \gqq{uberspace.de}-Server erreichbar. Die Wahl auf \gqq{uberspace} als Host fiel aufgrund der geringen monatlichen Mietkosten für das Hosting und die Servernutzung und der Unterstützung von MySQL. Die verwendeten Programmiersprachen umfassen Javascript und PHP. Dies ist der Tatsache geschuldet, dass viele meiner Kollegen und mein Ausbilder ein großes Know-How in diesen Sprachen haben und mich so effektiv unterstützen können. Des Weiteren ist für mich als Auszubildender in der Frontendentwicklung Javascript die Programmiersprache, in der ich mich am besten auskenne. Die Wahl auf PHP als serverseitige Sprache fiel aufgrund der großen Community im Vergleich zu NodeJS und der einfachen Datenbankanbindung an eine MySQL-Datenbank. NodeJS wird zwar für Single-Page-Anwendungen empfohlen, jedoch sollte man für rechenintensive Anwendungen, wie die Statistik-App eine ist, dann doch eher auf PHP zurückgreifen. Zudem gibt es von ePages auf der App-Entwicklungs-Webseite Tutorials zur App-Entwicklung in PHP und auch bereits eine eigene PHP-Klasse zur Anforderung von REST-Daten. Prinzipiell hätte man für diese Anwendung auch MongoDB als Datenbanksystem nehmen können. Das Vorhaben wurde aber mangels Erfahrung im Umgang damit verworfen.

\subsection{Architekturdesign}
\label{sec:Architekturdesign}

Die Frontendarchitektur der App wird unter Verwendung des Javascript-Frameworks ReactJS realisiert. ReactJS hat Ähnlichkeiten zu traditionellen \acs{MVC-Frameworks}, fokussiert sich jedoch besonders auf die Idee eine Website aus einzelnen interaktiven Komponenten zusammengebaut zu betrachten, die voneinander abhängig sein können (über Eltern-Kind-Beziehungen) und immer einen definierten (Daten-) Zustand haben, der sich dynamisch in Echtzeit ohne erneutes Laden der gesamten Website ändern kann.
Die Statistik-App soll genau diesen Ansprüchen entsprechen, da die einzelnen Diagramme, Tabellen und Interaktionsmöglichkeiten jeweils als eigenständige Komponenten bzw. Module aufgefasst werden können, die alle einen vordefinierten Zustand haben, der sich durch Interaktionen des Nutzers ändern kann. Dadurch fiel die Wahl schnell auf ReactJS als modernes Framework mit groß gewachsener Community und dessen Fokus auf zustandsbehaftete Anwendungen. Zudem wird es bereits innerhalb ePages für das neue Softwareprodukt verwendet, sodass auch hier schon ein hilfreiches Know-How vorhanden ist. Nicht zuletzt erleichtert ReactJS die modularisierte Programmierung mit Javascript, durch Einführung von Komponenten die alle von einer Basiskomponente abgeleitet sind. Außerdem unterstützt ReactJS die Verwendung von ECMAScript 6, wodurch nun auch mit Javascript eine vorgegaukelte objektorientierte Programmierung möglich ist.
Für die graphischen Darstellungen wird die Javascript-Bibliothek C3.js verwendet, welche kostenlos ist und von der extrem mächtigen D3.js-Bibliothek abgeleitet ist. C3.js bietet gerade den nötigen Umfang an Einstellungen, Darstellungsmöglichkeiten und Beispielen für die Statistik-App. Es gibt zwar wesentlich bessere und umfangreichere Bibliotheken für graphische Darstellungen, die meisten davon sind jedoch mindestens ab der kommerziellen Nutzung der App kostenpflichtig.
Serverseitig fiel die Entscheidung auf das PHP-Slim-Framework auf Anraten von Kollegen. Slim ist extra für Webanwendungen entwickelt worden, die mit vielen REST-Requests hantieren müssen und bietet einen schnellen und einfachen Einstieg in diese Grundfunktionalitäten, auf die es hier serverseitig für die App ankommt.
Die Kommunikation zwischen Frontend- und Backendskripten findet mittels AJAX statt.
Um die App ansatzweise \gqq{responsive} zu gestalten, wird das CSS-Framework Materialize \footnote{\url{http://materializecss.com/}} verwendet. Es bietet darüber hinaus vorgefertigte Designs und Javascript-Interaktionen für Navigationsleisten, Menüs, Buttons etc., welche in der App dann auch verwendet werden.

\subsection{Entwurf der Benutzeroberfläche}
\label{sec:Benutzeroberflaeche} 

\subsection{Datenmodell}
\label{sec:Datenmodell}
Das verwendete Datenmodell ist in \Abbildung{erm2} dargestellt. Dort sind insbesondere die Beziehungen und Kardinalitäten zwischen den Entitäten Shop, Kunde, Bestellung, Artikel, Kundenadresse usw. dargestellt.  
\begin{figure}[htb]
\centering
\includegraphicsKeepAspectRatio{erm2.png}{1.0}
\caption{ER-Modell: Beziehungen und Kardinalitäten zwischen den Entitäten}
\label{fig:erm2}
\end{figure}
\subsection{Geschäftslogik}
\label{sec:Geschaeftslogik}

Der Kern der Geschäftslogik liegt in den Programmabläufen der Frontendskripte, die abhängig von den Benutzerinteraktionen mit der App beim Client in den einzelnen React-Komponenten ablaufen und gegebenfalls AJAX-POST-Request zur (Daten-) Zustandsaktualisierung an den Server senden, bei dem Skripte für entsprechende Datenbankabfragen ablaufen. Exemplarisch sei hier die geplante Logik des Synchronisationsprozesses zwischen dem Onlineshop und der App anhand eines Sequenzdiagramms in \Anhang{app:sequenz} dargelegt.
Der Prozess soll durch ein Klickereignis auf den SYNC-Button in der Menüleiste ausgelöst werden, infolgedessen wird ein AJAX-Request an den Server gestellt, der sich wiederum per REST-calls die aktuellen Shopdaten holt und in die Datenbank einträgt.

Alle weiteren Klickereignisse, die der Nutzer über das Menü auslösen kann, sollen dann wenn notwendig nur noch Datenbankabfragen auslösen und keine weiteren REST-calls, um die REST-Schnittstelle zu schonen und die Abfrage- und Verarbeitungsprozesse zu beschleunigen.

\paragraph{Logik der Frontendarchitektur}
Die Statistik-App wird als Single-Page-Anwendung entwickelt, d.h. an keiner Stelle der App wird ein Neuladen der Seite ausgelöst. Dies wird durch React-Komponenten realisiert, die über Eltern-Kind-Beziehungen miteinander verknüpft sind. Dabei besitzt jede Komponente einen (Daten-)Zustand, der konstant oder veränderlich sein kann je nachdem, ob Daten aktualisiert werden oder nicht. Die Aktualisierung des Datenzustands findet ausschließlich asynchron statt, um während dieses Vorgangs die Benutzbarkeit der App nicht einzuschränken. Zusätzlich zum Zustand können jeder Kindkomponente von der Elternkomponente sog. \gqq{props} (properties/Eigenschaften) übergeben werden ähnlich zu dem Vererbungsprozess in der objektorientierten Programmierung bei abgeleiteten Klassen mit dem Unterschied, dass die Kindkomponente nicht automatisch alle Zustandsattribute und Methoden der Elternkomponent erbt, sondern nur jene, die ihr direkt zur Verwendung zugewiesen bzw. weitergereicht wurden, d.h. die übergebenen props können sowohl Zustandsattribute wie auch Methoden der Elternkomponente sein.

Der architektonische Frontendaufbau über die React-Komponenten und deren Abhängigkeiten zueinander ist in \Anhang{app:classes} dargestellt. 
Ähnlich zu einem UML-Klassendiagramm zeigen die Pfeile in Richtung der Elternkomponente, von der \gqq{geerbt} wird. Die Basiskomponente, von der alle anderen Komponenten aufgerufen werden, ist die \textbf{RoutedContent}-Komponente, in der über die React-Router-Komponente/Bibliothek die URL-Routen zu den anderen Basis-App-Komponenten definiert werden. Jede Komponente in der Abbildung besitzt einen Eintrag \textbf{state attributes}, der alle Zustandsdaten, falls vorhanden, auflistet. Diese Daten definieren den Zustand der Komponente. Zusätzlich werden alle in den Komponenten implementierten Methoden aufgelistet, die als props auch an Kindkomponenten weitergereicht werden können. Falls eine Komponente auch props übergeben bekommt, so werden diese zusätzlich auch noch aufgelistet.

\subsection{Maßnahmen zur Qualitätssicherung}
\label{sec:Qualitaetssicherung}

Um die Umsetzung aller Qualitätsanforderungen aus \ref{sec:Qualitaetsanforderungen} abzusichern, werden ständig händische Entwicklertests in verschiedenen Browsern durchgeführt, die darauf ausgerichtet sind, die Gesamtfunktionalität der App bei jeder neuen Implementierung oder Code-Änderung nicht zu gefährden. Zuletzt wird ein erfahrener teaminterner Mitarbeiter aus der QA-Abteilung damit beauftragt alle genannten Qualitätsanforderungen durchzutesten. Automatische Tests sind für dieses Projekt nicht vorgesehen.
