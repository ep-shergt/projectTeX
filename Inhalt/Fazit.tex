% !TEX root = ../Projektdokumentation.tex
\section{Fazit} 
\label{sec:Fazit}

\subsection{Soll-/Ist-Vergleich}
\label{sec:SollIstVergleich}

Abgesehen von den in \ref{sec:AbweichungenProjektantrag} erwähnten Abweichungen, die fehlende Artikelstatistiken und multiple Shopregistrierungen betreffen wurden alle anderen Sollvorgaben erfüllt und die App läuft in allen angestrebten Anforderungen stabil und korrekt.

In dem jetzigen Zustand ist die App noch nicht \gqq{deploymentfähig}, weshalb sie auch dem Produkt-Management zur Sichtung noch nicht vorgelegt wurde. Erst nach dem Beheben der Mängel, die sich aus dem Code-Review und dem QA-Testprotokoll ergaben, kann das auch das Feedback vom Produkt-Management gesucht werden.

\paragraph{Vergleich mit der Zeitplanung}
Summa summarum konnte der geplante Zeitrahmen für das Projekt eingehalten werden. Ein Vergleich ist im \Anhang{app:ZeitplanungReal} zu finden. Die einzigen größeren Abweichungen betreffen einen wesentlichen Mehraufwand bei der Erstellung aller notwendigen php-Skripte und einem reduzierten Aufwand bei der Erstellungen der notwendigen Javascript-UI-Interaktionen. Dies ist der geringeren Erfahrung im Umgang mit PHP geschuldet.

\subsection{Lessons Learned}
\label{sec:LessonsLearned}

Der Umgang mit neuen Frameworks (hier ReactJS) erfordert im Vorfeld ein hohes Maß an Recherche und Einarbeitung genauso wie die Verwendung eher unbekannter Programmiersprachen (hier PHP). Dies muss in der Zeitplanung eines Projekts unbedingt berücksichtigt werden, da der Mehraufwand, der daraus erwuchs, unterschätzt wurde. Außerdem muss mehr Augenmerk auf die \gqq{Definition of Ready}-Anforderungen gelegt werden bevor mit dem Projekt überhaupt begonnen werden kann, sodass Abweichungen wie in \ref{sec:AbweichungenProjektantrag} nicht erst überraschend während des Projektdurchführung auftreten, sondern im Vorfeld als bekannte Hinderlichkeiten in die Projektplanung miteinfließen.

Des Weiteren wäre es im Nachhinein schlauer gewesen, das Feedback aus der QA-Abteilung schon viel früher noch während der Entwicklung zu suchen genauso wie das Code-Review. Das legt den Schluss nahe, dass das Wasserfallmodell als Vorgangsmodell für solch ein umfangreiches Projekt eher unpassend und entweder durch das Spiralmodell oder agile Entwicklungszyklen ersetzt werden sollte.


\subsection{Ausblick}
\label{sec:Ausblick}

Die Entwicklung der App ist noch lange nicht abgeschlossen. Zunächst müssen die angesprochenen Mängel aus dem Code-Review und  dem Testprotokoll der QA-Abteilung beseitigt werden. Danach müssen die fehlenden Anforderungen aus Abschnitt \ref{sec:AbweichungenProjektantrag} noch implementiert werden. Daraus ergibt sich auch eine Änderung der Menüführung bei der Benutzung der App für den Nutzer, da so viele Statistiken nicht mehr auf einer Seite angezeigt werden können, sondern über das Menü dynamisch auf die Seite geladen werden sollten. Dabei sollte dann auch über Responsivität und die Darstellung der App auf mobilen Geräten nachgedacht und Nachbesserungen betrieben werden, da ePages mobile Geräte generell unterstützt. Schließlich müssen Bezahlmodalitäten bestimmt und eine Bezahlanbindung (Kreditkarte, Überweisung, PayPal etc.) geschaffen werden. Parallel dazu müssen AGBs formuliert und das Impressum erstellt werden. Zum Schluss sollten auch Sicherkonzepte überprüft und angepasst werden.
