% !TEX root = ../Projektdokumentation.tex
\section{Projektplanung} 
\label{sec:Projektplanung}


\subsection{Projektphasen}
\label{sec:Projektphasen}

Die Gesamtprojektbearbeitungszeit ist auf 70 Stunden festgelegt. Der Start des Projekts ist der 10.10.2016 und der Abschluss ist der 30.11.2016. Die 70 Stunden werden auf den Zeitraum relativ gleichmäßig verteilt, sodass mindestens halbtäglich noch den Tagesgeschäften nachgegangen und das Team unterstützt werden kann. Außerdem muss gewährleistet werden, dass ich trotz des Projektes an den wichtigsten Scrum- und Team-Meetings teilnehmen kann, die die Arbeitszeit regelmäßig und unregelmäßig unterbrechen.

Tabelle~\ref{tab:Zeitplanung} zeigt die grobe Zeitplanung.
\tabelle{Zeitplanung}{tab:Zeitplanung}{ZeitplanungKurz}\\
Eine detailliertere Zeitplanung findet sich im \Anhang{app:Zeitplanung}.


\subsection{Abweichungen vom Projektantrag}
\label{sec:AbweichungenProjektantrag}

Eine wesentliches Ziel der App sollte die Darstellung statistischer Analysen verkaufter Artikel des Onlineshops sein, der die App nutzt. Während der Projektdurchführung wurde jedoch festgestellt, dass die genutzte \acs{PHP}-Order-Klasse für die \acs{REST}-Calls noch keine Möglichkeit bietet Artikel bzw. Produkte aus den Shopbestellungen zu extrahieren, obwohl im Vorfeld fälschlicherweise davon ausgegangen wurde, dass diese Möglichkeit besteht\footnote{\url{https://github.com/ePages-de/epages-rest-php/blob/master/src/shopobjects/order/Order.class.php}}. Aus Zeitgründen war es bisher weder mir noch dem verantwortlichen Entwickler möglich die Klasse dahingehend zu erweitern, sodass auf Artikelstatistiken gänzlich verzichtet wurde. 

Des Weiteren wurde die Registrierungsmöglichkeit für Neukunden der App bisher nur soweit aufbereitet, dass alle Kunden automatisch den ePages-Developer-Test-Shop zugewiesen bekommen, da die Anbindung an echte Onlineshops noch gar nicht getestet werden kann. Dazu ist erst das Einverständnis des Produktmanagements nach Sichtung der App notwendig.

Außerdem wurde auf die Erstellung eines Pflichtenheftes verzichtet, da kein echter Auftraggeber existiert, sondern das Projekt selbst motiviert ist. Alle Anforderungen an die App werden schon im Lastenheft ausformuliert und würden sich im Pflichtenheft nur doppeln. So kann die Zeit zur Erstellung des Pflichtenheftes eingespart und zur Implementierung verwendet werden. 



\subsection{Ressourcenplanung}
\label{sec:Ressourcenplanung}

An Ressourcen ist ein PC-Arbeitsplatz (Schreibtisch, Rollcontainer, Schreibtischstuhl, Schreibtischlampe) in einem Firmenbüro vonnöten. Dazu kommt an Hardware ein \acs{LAN}- und \acs{WLAN}-fähiger COREi7-Laptop mit 16 GByte Arbeitsspeicher und mindestens 100 GByte großer Festplatte sowie zwei Monitore, eine Maus und ein Headset. An Software wird ein Windows 10 Betriebssystem, Sublime Text 3 als Text-Editor, ein Internetbrowser (Google Chrom und Firefox), Putty als \acs{SSH}-Client, Win\acs{SCP} als \acs{SFTP} und \acs{FTP}-Client-Software, ein externer Server, auf dem die App \gqq{gehostet} wird bereitgestellt von \textit{uberspace.de}, phpMyAdmin \footnote{\url{https://www.phpmyadmin.net/}} zur Verwaltung der Datenbank, HipChat \footnote{\url{https://www.hipchat.com/}} als internes Chattool der Firma, JIRA \footnote{\url{https://www.atlassian.com/software/jira}} von Atlassian als Verwaltungssoftware der agilen Entwicklungsprozesse unter \acs{Scrum} sowie Confluence von Atlassian als interne Kollaborationssoftware für Teams. Dazu kommt die Verwendung von \acs{Git} und \acs{GitHub} zur Versionierungskontrolle. Darüberhinaus werden auch personelle Ressouren beansprucht, die einen Backendentwickler zum Einrichten des epages Developer-Shops und des Slim-Frameworks sowie eine \acs{QA}-Kollegin zum Testen der App-Funktionen und meinen Ausbilder zur generellen Überwachung meines Projektes umfassen.


\subsection{Entwicklungsprozess}
\label{sec:Entwicklungsprozess}

Für die Entwicklung wird das Wasserfallmodell verwendet, in dem bestimmte Meilensteine gesetzt und abgeschlossene Phasen definiert werden können. Dadurch werden insbesondere eine gut definierte und ausgereifte Planungs- und Entwurfsphase ermöglicht, die zur eigentlichen Implementierung notwendig sind.