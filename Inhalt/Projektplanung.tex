% !TEX root = ../Projektdokumentation.tex
\section{Projektplanung} 
\label{sec:Projektplanung}


\subsection{Projektphasen}
\label{sec:Projektphasen}

Die Gesamtprojektbearbeitungszeit ist auf 70 Stunden festgelegt. Der Start des Projekts ist der 10.10.2016 und der Abschluss ist der 30.11.2016. Die 70 Stunden werden auf den Zeitraum relativ gleichmäßig verteilt, sodass mindestens halbtäglich noch den Tagesgeschäften nachgegangen und das Team unterstützt werden kann. Außerdem muss gewährleistet werden, dass ich trotz des Projektes an den wichtigsten Scrum- und Team-Meetings teilnehmen kann, die die Arbeitszeit regelmäßig und unregelmäßig unterbrechen.

\paragraph{Beispiel}
Tabelle~\ref{tab:Zeitplanung} zeigt ein Beispiel für eine grobe Zeitplanung.
\tabelle{Zeitplanung}{tab:Zeitplanung}{ZeitplanungKurz}\\
Eine detailliertere Zeitplanung findet sich im \Anhang{app:Zeitplanung}.


\subsection{Abweichungen vom Projektantrag}
\label{sec:AbweichungenProjektantrag}

\begin{itemize}
	\item Sollte es Abweichungen zum Projektantrag geben (\zB Zeitplanung, Inhalt des Projekts, neue Anforderungen), müssen diese explizit aufgeführt und begründet werden.
\end{itemize}


\subsection{Ressourcenplanung}
\label{sec:Ressourcenplanung}

An Ressourcen ist ein PC-Arbeitsplatz (Schreibtisch, Rollcontainer, Schreibtischstuhl, Schreibtischlampe) in einem Firmenbüro vonnöten. Dazu kommt an Hardware ein LAN- und WLAN-fähiger COREi7-Laptop mit 16 GByte Arbeitsspeicher und mindestens 100 GByte großer Festplatte sowie zwei Monitore, eine Maus und ein Headset. An Software wird ein Windows 10 Betriebssystem, Sublime Text 3 als Text-Editor, ein Internetbrowser (Google Chrom und Firefox), Putty als \acs{SSH}-Client, Win\acs{SCP} als \acs{SFTP} und \acs{FTP}-Client-Software ein externer Server, auf dem die App \gqq{gehostet} wird bereitgestellt von \gqq{uberspace.de}, phpMyAdmin zur Verwaltung der Datenbank, HipChat als internes Chattool der Firma, JIRA von Atlassian als Verwaltungssoftware der agilen Entwicklungsprozesse unter Scrum sowie Confluence von Atlassian als interne Kollaborationssoftware für Teams. Dazu kommt die Verwendung von \acs{Git} und \acs{GitHub} zur Versionierungskontrolle. Darüberhinaus werden auch personelle Ressouren beansprucht, die einen Backendentwickler zum Einrichten des epages Developer-Shops und des Slim-Frameworks sowie eine \acs{QA}-Kollegin zum Testen der App-Funktionen und meinen Ausbilder zur generellen Überwachung meines Projektes umfassen.

\begin{itemize}
	\item Detaillierte Planung der benötigten Ressourcen (Hard-/Software, Räumlichkeiten \usw).
	\item \Ggfs sind auch personelle Ressourcen einzuplanen (\zB unterstützende Mitarbeiter).
	\item Hinweis: Häufig werden hier Ressourcen vergessen, die als selbstverständlich angesehen werden (\zB PC, Büro). 
\end{itemize}


\subsection{Entwicklungsprozess}
\label{sec:Entwicklungsprozess}

Für die Entwicklung wird das Wasserfallmodell verwendet, in dem bestimmte Meilensteine gesetzt und abgeschlossene Phasen definiert werden können. Dadurch werden insbesondere eine gut definierte und ausgereifte Planungs- und Entwurfsphase ermöglicht, die zur eigentlichen Implementierung notwendig sind. 
\begin{itemize}
	\item Welcher Entwicklungsprozess wird bei der Bearbeitung des Projekts verfolgt (\zB Wasserfall, agiler Prozess)?
\end{itemize}
