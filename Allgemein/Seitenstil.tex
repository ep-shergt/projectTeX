% !TEX root = ../Projektdokumentation.tex

% Seitenränder -----------------------------------------------------------------
\setlength{\topskip}{\ht\strutbox} % behebt Warnung von geometry
\geometry{a4paper,left=35mm,right=30mm,top=40mm,bottom=30mm}

\usepackage[
	automark, % Kapitelangaben in Kopfzeile automatisch erstellen
	% headsepline, % Trennlinie unter Kopfzeile
	ilines % Trennlinie linksbündig ausrichten
]{scrpage2}

\usepackage{fancyhdr}

\fancyhf{}
\fancyhead[C]{\thepage}
\pagestyle{fancy}

% redefine the plain pagestyle
\fancypagestyle{plain}{%
\fancyhf{} % clear all header and footer fields
\fancyhead[C]{\thepage} % except the center
}

% Kopf- und Fußzeilen ----------------------------------------------------------
%\pagestyle{headings}
% chapterpagestyle gibt es nicht in scrartcl
%\renewcommand{\chapterpagestyle}{scrheadings}
\clearscrheadfoot

% Kopfzeile
%\renewcommand{\headfont}{\normalfont} % Schriftform der Kopfzeile
%\ihead{\large{\textsc{\titel}}\\ \small{\untertitel} \\[2ex] %\textit{\headmark}}
%\chead{}
%\ohead{\includegraphics[scale=0.2]{\betriebLogo}}
%\setlength{\headheight}{15mm} % Höhe der Kopfzeile
%\setheadwidth[0pt]{textwithmarginpar} % Kopfzeile über den Text hinaus verbreitern (falls Logo den Text überdeckt)

% Fußzeile
\ifoot{\autorName}
\cfoot{}
\ofoot{\pagemark}

% Überschriften nach DIN 5008 in einer Fluchtlinie
% ------------------------------------------------------------------------------

% Abstand zwischen Nummerierung und Überschrift definieren
% > Schön wäre hier die dynamische Berechnung des Abstandes in Abhängigkeit
% > der Verschachtelungstiefe des Inhaltsverzeichnisses
%\newcommand{\headingSpace}{1.5cm}

% Abschnittsüberschriften im selben Stil wie beim Inhaltsverzeichnis einrücken
\renewcommand*{\othersectionlevelsformat}[3]{
  \makebox[\headingSpace][l]{#3\autodot}
}

% Für die Einrückung wird das Paket tocloft benötigt
%\cftsetindents{chapter}{0.0cm}{\headingSpace}
\cftsetindents{section}{0.0cm}{\headingSpace}
\cftsetindents{subsection}{0.0cm}{\headingSpace}
\cftsetindents{subsubsection}{0.0cm}{\headingSpace}
\cftsetindents{figure}{0.0cm}{\headingSpace}
\cftsetindents{table}{0.0cm}{\headingSpace}


% Allgemeines
% ------------------------------------------------------------------------------


\onehalfspacing % Zeilenabstand 1,5 Zeilen
\frenchspacing % erzeugt ein wenig mehr Platz hinter einem Punkt

% Schusterjungen und Hurenkinder vermeiden
\clubpenalty = 10000
\widowpenalty = 10000
\displaywidowpenalty = 10000

% Quellcode-Ausgabe formatieren
\lstset{numbers=left, numberstyle=\tiny, numbersep=5pt, breaklines=true}
\lstset{emph={square}, emphstyle=\color{red}, emph={[2]root,base}, emphstyle={[2]\color{blue}}}

\counterwithout{footnote}{section} % Fußnoten fortlaufend durchnummerieren
\setcounter{tocdepth}{\subsubsectionlevel} % im Inhaltsverzeichnis werden die Kapitel bis zum Level der subsubsection übernommen
\setcounter{secnumdepth}{\subsubsectionlevel} % Kapitel bis zum Level der subsubsection werden nummeriert

% Aufzählungen anpassen
\renewcommand{\labelenumi}{\arabic{enumi}.}
\renewcommand{\labelenumii}{\arabic{enumi}.\arabic{enumii}.}
\renewcommand{\labelenumiii}{\arabic{enumi}.\arabic{enumii}.\arabic{enumiii}}

% Tabellenfärbung:
\definecolor{heading}{rgb}{0.64,0.78,0.86}
\definecolor{odd}{rgb}{0.9,0.9,0.9}
